\documentclass[a4paper,11pt]{scrartcl}
\usepackage[utf8]{inputenc}
\usepackage[english]{babel}
\usepackage{amssymb}
\usepackage{amsmath}
\usepackage{amsthm}

\newcommand{\R}{\mathbb{R}}
\newcommand{\N}{\mathbb{N}}
\newcommand{\Q}{\mathbb{Q}}

\title{Seminarberbeit aus Numerik oder so}
\subtitle{Parallel sparse Zeug}
\author{Bernd Schwarzenbacher, Daniel Herold}

\begin{document}


\maketitle
\tableofcontents

\pagebreak


The goal of the AMG-Preconditioner is to combine severel bad preconditioners to
get a good preconditioner.
In every iteration Gauß-Seidel preconditioners are used on each level back and
forth. There lies a great potential to save time.
The normal matrix vector multiplication is easely programmed parallel. Due to
the sparse data structure it's although harder to make the transposed matrix
vector multiplication parallel.

%\begin{lstlisting}

  %void my_AMG_H1 :: Mult (const ngla::BaseVector & b,
                          %ngla::BaseVector & x) const
  %{
    %static Timer timer("H1-AMG::Mult");
    %RegionTimer reg (timer);

    %if (inv)
    %{
      %x = (*inv) * b;
      %return;
    %}

    %auto residuum = pmat->CreateVector();
    %auto coarse_x = coarsemat->CreateVector();
    %auto coarse_residuum = coarsemat->CreateVector();

    %x = 0;
    %jacobi->GSSmooth (x, b);

    %if (recAMG)
    %{
      %residuum = b - (*pmat) * x;
      %coarse_residuum = ngla::Transpose (*prol) * residuum;

      %recAMG->Mult(coarse_residuum, coarse_x);

      %x += (*prol) * coarse_x;
    %}

    %jacobi->GSSmoothBack (x, b);
  %}
%\end{lstlisting} 

\end{document}

