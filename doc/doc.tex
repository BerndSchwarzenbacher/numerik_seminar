\documentclass[a4paper,11pt]{scrartcl}
\usepackage{fullpage}
\usepackage[utf8]{inputenc}
\usepackage[english]{babel}
\usepackage{amssymb}
\usepackage{amsmath}
\usepackage{amsthm}
\usepackage{listings}

\newcommand{\R}{\mathbb{R}}
\newcommand{\N}{\mathbb{N}}
\newcommand{\Q}{\mathbb{Q}}

\title{seminararbeit aus numerik oder so}
\subtitle{parallel sparse zeug}
\author{Bernd Schwarzenbacher, Daniel Herold}

\begin{document}


\maketitle
\tableofcontents

\pagebreak

\section{Motivation}

To solve big systems of linear equations the conjugate gradient method is the
most used iterative method.
For faster convergence speed the condition of the problem is
adjusted with a so called preconditioner.
The original problem $b-Ax=0$ is substituted by the preconditioned problem
$C^{-1} (b-Ax) = 0$ with preconditioner $C$.
A simple choice is the Gauss-Seidel preconditioner.

The goal of the AMG (Algebraic Multi Grid) preconditioner is to combine several
bad preconditioners to get a good preconditioner. To achieve this, the original
matrix is coarsened recursively to smaller matrices (hence Multi Grid).
Weighing the relative strength of the connecting vertices, a prolongation matrix
is computed to project the problem on a coarser space. To determine which
vertices should collapse, only the information in the system matrix is needed
(hence Algebraic).
On the coarser levels, big errors in the solution are faster corrected.
In each iteration Gauß-Seidel preconditioners are used.
Here lies a great potential to save time.

On the coarsest level the problem is solved directly with the inverse matrix.
(Line 4-8).
Otherwise we do a Gauß-Seidel smoothing forward in line 15.
And if there is a coarser level, we calculate the current residuum and
prolongate it to the coarser level (line 20). Here the transposed
matrix vector multiplication is used. Then the same routine is called
recursevily at line 22 with the coarse residuum and the coarse solution. The
coarse solution then is added back onto the fine solution.

\lstset{language=C++, numbers=left, captionpos=b,
        caption={Multiplication Method of the AMG-Preconditioner}}
\begin{lstlisting}
void my_AMG_H1 :: Mult (const ngla::BaseVector & b,
                        ngla::BaseVector & x) const
{
  if (inv)
  {
    x = (*inv) * b;
    return;
  }

  auto residuum = pmat->CreateVector();
  auto coarse_x = coarsemat->CreateVector();
  auto coarse_residuum = coarsemat->CreateVector();

  x = 0;
  jacobi->GSSmooth (x, b);

  if (recAMG)
  {
    residuum = b - (*pmat) * x;
    coarse_residuum = ngla::Transpose (*prol) * residuum;

    recAMG->Mult(coarse_residuum, coarse_x);

    x += (*prol) * coarse_x;
  }

  jacobi->GSSmoothBack (x, b);
}
\end{lstlisting}

\section{Sparsematrices}

Most of the matrices are stored in the ccs-format (compressed column storage).
 The commands {\em firsti[k]} and {\em lasti[k]} provide the position of the
 first and last entry of the row k in the indexvector. One
 iteration through one row ist very easy to provide.

\lstset{language=C++, numbers=left, captionpos=b,
        caption={Row-iteration of sparsematrices }}
\begin{lstlisting}
for (int j = firsti[k]; j < lasti[k]; ++j)
      {...}

\end{lstlisting}

The difficulty with sparsematrices is the iteration among any other order as
the next section will show.

\section{Matrix vector multiplication}

The multiplication of a sparsematrix and a given vector is very simple.
The entry k of the outcoming vector is given by the summation of the products
of the matrixentries in the row k with the entries of the vector. In other
words one entry of the outcoming vector depends only on one row of the matrix
and vice versa.

One of the difficulties of parallelization is given by multiple writing
performances on a single variable at the same time. In the case of a simple
 matrix vector multiplication this is no big deal to handle: Each row has to be
executed by a single task, there is no limitation in which order which task
 calculates 'his' vectorentry.

\lstset{language=C++, numbers=left, captionpos=b,
        caption={Matrix vector multiplication with parallel for section}}
\begin{lstlisting}
void MultAdd3 (double s, const BaseVector & x, BaseVector & y) const
  {
    static Timer timer("SparseMatrix::MultAdd-Parallel");
    RegionTimer reg (timer);
    timer.AddFlops(this->nze);
    FlatVector<double> fx = x.FV<double> ();
    FlatVector<double> fy = y.FV<double> ();

#pragma omp parallel for
    for (int i = 0; i < this->Height(); ++i)
    {
      int first = firsti [i];
      int last  = firsti [i+1];

      for (int j = first; j < last; ++j)
      {
        fy(i) += s * data[j] * fx(colnr[j]);
      }
    }
  }

\end{lstlisting}

\subsection{Transposed matrix vector multiplication}

To provide the multiplication of the transposed of a matrix and a given vector
 a sequential script is written in a similar way.

\lstset{language=C++, numbers=left, captionpos=b,
        caption={Transposed vector multiplication, sequential}}
\begin{lstlisting}
void TranMultAdd1 (double s, const BaseVector & x, BaseVector & y) const
  {
    static Timer timer("SparseMatrix::TranMultAdd-Sequential");
    RegionTimer reg (timer);
    timer.AddFlops(this->nze);
    FlatVector<double> fx = x.FV<double> ();
    FlatVector<double> fy = y.FV<double> ();

    for (int i = 0; i < this->Height(); ++i)
    {
      int first = firsti [i];
      int last  = firsti [i+1];

      for (int j = first; j < last; ++j)
      {
        fy(colnr[j]) += s * data[j] * fx(i);
      }

    }
  }


\end{lstlisting}

The difference is the dependency of a single outcoming entry. It is calculated
 by the entries of a column of the matrix - but because of the storage it is not
possible to iterate through one column in an efficient way. If the algorithm
has to iterate through one line of the matrices, simple parallelization would
probably cause issues in some outcoming entries due to multiple writing
performances at the same time.

Some ideas to solve this problem are:

\begin{itemize}

\item atomic writing process (easy to program, but very inefficient)
\item atomic writing process with dynamic task partioning
\item atomic writing process with static thread seperation
\item matrix coloring without any atomic processes

\end{itemize}

All of this solutions efficiencies depend on the number of entries in the matrix.
Less entries and a larger matrix result in a higher amount of calculations per
 time where a sequential solving of a matrix with a lot of nonzeroelements is
 faster than any parallel solution.

The atomic version is very straight forward:

\lstset{language=C++, numbers=left, captionpos=b,
        caption={Transposed vector multiplication, parallel for with atomic}}
\begin{lstlisting}
void TranMultAdd2 (double s, const BaseVector & x, BaseVector & y) const
  {
    static Timer timer("SparseMatrix::TranMultAdd-Parallel-For");
    RegionTimer reg (timer);
    timer.AddFlops(this->nze);
    FlatVector<double> fx = x.FV<double> ();
    FlatVector<double> fy = y.FV<double> ();

#pragma omp parallel
    {
#pragma omp for
      for (int i = 0; i < this->Height(); ++i) {
        int first = firsti [i];
        int last  = firsti [i+1];

        for (int j = first; j < last; ++j)
        {
#pragma omp atomic
          fy(colnr[j]) += s * data[j] * fx(i);
        }

      }
    }
  }

\end{lstlisting}

The bottleneck of this algorithm is line 19, because every task has to pass this
line and they lock each other.

The dynamic task partitioning solves another problem of this code. {\em OMP for}
 splits the for-loop into static iterations for each task. Due to the amount
 of calculations each task has to perform, some will be finished while others
are still working. This relates to the specific structur of the matrix. The
 number of nonzeroelements in euch row can be very different in AMG.

The time that is used for the dynamic assignment can be retrieved by the fair
partitioning of the tasks. The number 100 demands, that not every single loop
 will be assigned dynamically but every 100 of them.

\lstset{language=C++, numbers=none, captionpos=b,
        caption={Dynamic for loop}}
\begin{lstlisting}
...
#pragma omp for schedule (dynamic, 100)
...
\end{lstlisting}

The balancing option performs this action in a similar way, but not during the
for-loop itself but first of it. The time for the calculation of each line
depends (as mentioned) on the number of nonzeroelements in this row. With a
simple algorithm it is possible to  determine this number for each row of the
 and split the loop due to this partitioning.

\lstset{language=C++, numbers=left, captionpos=b,
        caption={Transposed vector multiplication, static balancing}}
\begin{lstlisting}
void TranMultAdd3 (double s, const BaseVector & x, BaseVector & y) const
  {
    static Timer timer("SparseMatrix::TranMultAdd-Balancing");
    RegionTimer reg (timer);
    timer.AddFlops(this->nze);
    FlatVector<double> fx = x.FV<double> ();
    FlatVector<double> fy = y.FV<double> ();

    int height = this->Height();

    Array<int> thread_seperation;
#pragma omp parallel //shared(thread_seperation)
    {
      int num_threads = omp_get_num_threads();

#pragma omp single
      {
        thread_seperation = Array<int>(num_threads+1);
        int seperation_step = ceil(this->nze / num_threads);

        int thread_i = 1;
        for (int row = 0; row < height; ++row)
        {
          if (firsti[row] >= thread_i * seperation_step)
          {
            thread_seperation[thread_i] = row;
            ++thread_i;
          }
        }
        thread_seperation[0] = 0;
        thread_seperation[num_threads] = height;
      }

#pragma omp for
      for (int thread_i = 1; thread_i <= num_threads; ++thread_i)
      {
        for (int i = thread_seperation[thread_i-1];
             i < thread_seperation[thread_i]; ++i)
        {
          int first = firsti [i];
          int last  = firsti [i+1];

          for (int j = first; j < last; ++j)
          {
#pragma omp atomic
            fy(colnr[j]) += s * data[j] * fx(i);
          }
        }
      }
    }
  }

\end{lstlisting}

Every method still has the same problem with the atomic operation in its
essential part. To avoid this, we have to make sure that multiple tasks do not
write on the same entry of the outcoming vector at the same time. To solve this,
 we will "color" each row of the matrix. Two group of rows will get the same
color, if they do not write on the same entry. This is equivalent to not having
any columns with nonzeroelements in common. With this information we can iterate through the different colors sequential but parallelize every loop through
 a group of same-colored-rows, because they will not interfere.

\lstset{language=C++, numbers=left, captionpos=b,
        caption={Coloring of a matrix for transposed vector multiplication}}
\begin{lstlisting}
void Coloring ()
  {
    static Timer timer("SparseMatrix::Coloring");
    RegionTimer reg (timer);
    timer.AddFlops(this->nze);

    int height = this->Height();
    int width = this->Width();

    Array<int> row_color(height);
    row_color = -1;

    int maxcolor = 0;
    int basecol = 0;

    Array<unsigned int> mask(width);

    int found = 0;

    do
    {
      mask = 0;

      for (int row = 0; row < height; ++row)
      {
        if (row_color[row] >= 0) continue;

        int first = firsti [row];
        int last  = firsti [row+1];

        unsigned check = 0;
        for (int i = first; i < last; ++i)
          check |= mask[colnr[i]];

        if (check != UINT_MAX) // 0xFFFFFFFF)
        {
          found++;
          unsigned checkbit = 1;
          int color = basecol;
          while (check & checkbit)
          {
            color++;
            checkbit *= 2;
          }

          row_color[row] = color;
          if (color > maxcolor) maxcolor = color;

          for (int i = first; i < last; ++i)
            mask[colnr[i]] |= checkbit;
        }
      }

      basecol += 8*sizeof(unsigned int); // 32;
    }
    while (found < height);

    Array<int> cntcol(maxcolor+1);
    cntcol = 0;
    for (int row = 0; row < height; ++row)
      ++cntcol[row_color[row]];

    coloring_ = table<int>(cntcol);

    cntcol = 0;
    for (int row = 0; row < height; ++row)
      coloring_[row_color[row]][cntcol[row_color[row]]++] = row;

    std::cout << "needed " << maxcolor+1 << " colors" << std::endl;
  }

\end{lstlisting}

\lstset{language=c++, numbers=left, captionpos=b,
        caption={transposed vector multiplication by coloring a matrix}}
\begin{lstlisting}
void TranMultAdd4 (double s, const BaseVector & x, BaseVector & y)
  {
    static Timer timer("SparseMatrix::TranMultAdd-Coloring");
    RegionTimer reg (timer);
    timer.AddFlops(this->nze);
    FlatVector<double> fx = x.FV<double> ();
    FlatVector<double> fy = y.FV<double> ();

    Coloring();

#pragma omp parallel
    {
      for (auto color : coloring_)
      {
#pragma omp for
        for (int i = 0; i < color.Size(); ++i)
        {
          int first = firsti [color[i]];
          int last  = firsti [color[i]+1];

          for (int j = first; j < last; ++j)
          {
            fy(colnr[j]) += s * data[j] * fx(color[i]);
          }
        }
      }

    }
  }


\end{lstlisting}

\section{Gauß-Seidel-Method}
The Gauß-Seidel-Method is similar to the Jacobi-Method. The difference is,
that it also uses the new information optioned by previous computations of
entries of the solution vector.
$$ x_{j} = \frac{1}{a_{jj}} (b_{j} - A_{j} x) = \frac{1}{a_{jj}} (b_{j} - $$
Where $A_{j}$ denotes the j-th Row of the matrix $A$.

In the end we implemented a coloring alghorithm.
Similar to the coloring method of matrix vector multiplication,
we compute entries of the solution vector parallel,
when this entries don't influence each other.
Because the matrix is symmetric it is sufficient to look at
$A_{ij}$ or $A_{ji}$, if this matrix entrie is zero,
then the two indices can be in the same color.
To optain symmetrie we cycle through the colors in reverse order, in
GSSmoothBack.


\end{document}
