\documentclass[a4paper,11pt]{scrartcl}
\usepackage{fullpage}
\usepackage[utf8]{inputenc}
\usepackage[english]{babel}
\usepackage{amssymb}
\usepackage{amsmath}
\usepackage{amsthm}
\usepackage{listings}

\newcommand{\R}{\mathbb{R}}
\newcommand{\N}{\mathbb{N}}
\newcommand{\Q}{\mathbb{Q}}

\title{Seminararbeit aus Numerik oder so}
\subtitle{Parallel sparse Zeug}
\author{Bernd Schwarzenbacher, Daniel Herold}

\begin{document}


\maketitle
\tableofcontents

\pagebreak

Preconditioning is used in the CG-Approximation Method. The goal of the
AMG-Preconditioner is to combine severel bad preconditioners to
get a good preconditioner. To achieve this, the original Matrix is coarsened
recursively to smaller matrices. Weighing the relative strength of the
connecting vertices through the matrix entries, a prolongation matrix is
computed to project the fine matrix on the next coarser space. On the coarser
levels, big errors in the solution are faster corrected.
In every iteration Gauß-Seidel preconditioners are used on each level back and
forth. There lies a great potential to save time.
The normal matrix vector multiplication is easely programmed parallel. Due to
the sparse data structure it's although harder to make the transposed matrix
vector multiplication parallel.

On the coarsest level the problem is solved directly with the inverse matrix.
(Line 7-11).
Otherwise we do a Gauß-Seidel smoothing forward in line 18.
And if there is a coarser level, we calculate the current residuum and
prolongate it to the coarser level (line 23). Here the transposed
matrix vector multiplication is used. Then the same routine is called
recursevily at line 25 with the coarse residuum and the coarse solution. The
coarse solution then is added back onto the fine solution.

\lstset{language=C++, numbers=left, captionpos=b,
        caption={Multiplication Method of the AMG-Preconditioner}}
\begin{lstlisting}
void my_AMG_H1 :: Mult (const ngla::BaseVector & b,
                        ngla::BaseVector & x) const
{
  static Timer timer("H1-AMG::Mult");
  RegionTimer reg (timer);

  if (inv)
  {
    x = (*inv) * b;
    return;
  }

  auto residuum = pmat->CreateVector();
  auto coarse_x = coarsemat->CreateVector();
  auto coarse_residuum = coarsemat->CreateVector();

  x = 0;
  jacobi->GSSmooth (x, b);

  if (recAMG)
  {
    residuum = b - (*pmat) * x;
    coarse_residuum = ngla::Transpose (*prol) * residuum;

    recAMG->Mult(coarse_residuum, coarse_x);

    x += (*prol) * coarse_x;
  }

  jacobi->GSSmoothBack (x, b);
}
\end{lstlisting}

\section{Sparsematrices}

Most of the matrices are stored in the ccs-format (compressed column storage).
 The commands {\em firsti[k]} and {\em lasti[k]} provide the position of the
 first and last entry of the row k in the indexvector. One
 iteration through one row ist very easy to provide.

\begin{lstlisting}

for (int j = firsti[k]; j < last[k]; ++j)
      {...}

\end{lstlisting}

The difficulty with sparsematrices is the iteration among any other order as 
the next section will show.

\section{Matrix vector multiplication}

The multiplication of a sparsematrix and a given vector is very simple. 
The entry k of the outcoming vector is given by the summation of the products 
of the matrixentries in the row k with the entries of the vector. In other 
words one entry of the outcoming vector depends only on one row of the matrix 
and vice versa.

One of the difficulties of parallelization is given by multiple writing 
performances on a single variable at the same time. In the case of a simple
 matrix vector multiplication this is no big deal to handle: Each row has to be
executed by a single task, there is no limitation in which order which task
 calculates 'his' vectorentry.

\begin{lstlisting}

  void MultAdd3 (double s, const BaseVector & x, BaseVector & y) const
  {
    static Timer timer("SparseMatrix::MultAdd-Parallel");
    RegionTimer reg (timer);
    timer.AddFlops(this->nze);
    FlatVector<double> fx = x.FV<double> ();
    FlatVector<double> fy = y.FV<double> ();

#pragma omp parallel for
    for (int i = 0; i < this->Height(); ++i)
    {
      int first = firsti [i];
      int last  = firsti [i+1];

      for (int j = first; j < last; ++j)
      {
        fy(i) += s * data[j] * fx(colnr[j]);
      }
    }
  }

\end{lstlisting}

\subsection{Transposed matrix vector multiplication}

To provide the multiplication of the transposed of a matrix and a given vector
 a sequential script is written in a similar way.

\begin{lstlisting}

	void TranMultAdd1 (double s, const BaseVector & x, BaseVector & y) const
  {
    static Timer timer("SparseMatrix::TranMultAdd-Sequential");
    RegionTimer reg (timer);
    timer.AddFlops(this->nze);
    FlatVector<double> fx = x.FV<double> ();
    FlatVector<double> fy = y.FV<double> ();

    for (int i = 0; i < this->Height(); ++i)
    {
      int first = firsti [i];
      int last  = firsti [i+1];

      for (int j = first; j < last; ++j)
      {
        fy(colnr[j]) += s * data[j] * fx(i);
      }

    }
  }


\end{lstlisting}

The difference is the dependency of a single outcoming entry. It is calculated
 by the entries of a column of the matrix - but because of the storage it is not
possible to iterate through one column in an efficient way. If the algorithm 
has to iterate through one line of the matrices, simple parallelization would
probably cause issues in some outcoming entries due to multiple writing
performances at the same time. 

Some ideas to solve this problem are:

\begin{itemize}

\item atomic writing process (easy to program, but very inefficient)
\item atomic writing process with dynamic task partioning
\item atomic writing process with static thread seperation
\item matrix coloring without any atomic processes

\end{itemize} 

All of this solutions efficiencies depend on the number of entries in the matrix.
Less entries and a larger matrix result in a higher amount of calculations per
 time where a sequential solving of a matrix with a lot of nonzeroelements is
 faster than any parallel solution.



\end{document}

